% Options for packages loaded elsewhere
\PassOptionsToPackage{unicode}{hyperref}
\PassOptionsToPackage{hyphens}{url}
%
\documentclass[
]{book}
\usepackage{amsmath,amssymb}
\usepackage{lmodern}
\usepackage{iftex}
\ifPDFTeX
  \usepackage[T1]{fontenc}
  \usepackage[utf8]{inputenc}
  \usepackage{textcomp} % provide euro and other symbols
\else % if luatex or xetex
  \usepackage{unicode-math}
  \defaultfontfeatures{Scale=MatchLowercase}
  \defaultfontfeatures[\rmfamily]{Ligatures=TeX,Scale=1}
\fi
% Use upquote if available, for straight quotes in verbatim environments
\IfFileExists{upquote.sty}{\usepackage{upquote}}{}
\IfFileExists{microtype.sty}{% use microtype if available
  \usepackage[]{microtype}
  \UseMicrotypeSet[protrusion]{basicmath} % disable protrusion for tt fonts
}{}
\makeatletter
\@ifundefined{KOMAClassName}{% if non-KOMA class
  \IfFileExists{parskip.sty}{%
    \usepackage{parskip}
  }{% else
    \setlength{\parindent}{0pt}
    \setlength{\parskip}{6pt plus 2pt minus 1pt}}
}{% if KOMA class
  \KOMAoptions{parskip=half}}
\makeatother
\usepackage{xcolor}
\IfFileExists{xurl.sty}{\usepackage{xurl}}{} % add URL line breaks if available
\IfFileExists{bookmark.sty}{\usepackage{bookmark}}{\usepackage{hyperref}}
\hypersetup{
  pdftitle={R-Blitzkurs},
  pdfauthor={Sebastian Sauer},
  hidelinks,
  pdfcreator={LaTeX via pandoc}}
\urlstyle{same} % disable monospaced font for URLs
\usepackage{longtable,booktabs,array}
\usepackage{calc} % for calculating minipage widths
% Correct order of tables after \paragraph or \subparagraph
\usepackage{etoolbox}
\makeatletter
\patchcmd\longtable{\par}{\if@noskipsec\mbox{}\fi\par}{}{}
\makeatother
% Allow footnotes in longtable head/foot
\IfFileExists{footnotehyper.sty}{\usepackage{footnotehyper}}{\usepackage{footnote}}
\makesavenoteenv{longtable}
\usepackage{graphicx}
\makeatletter
\def\maxwidth{\ifdim\Gin@nat@width>\linewidth\linewidth\else\Gin@nat@width\fi}
\def\maxheight{\ifdim\Gin@nat@height>\textheight\textheight\else\Gin@nat@height\fi}
\makeatother
% Scale images if necessary, so that they will not overflow the page
% margins by default, and it is still possible to overwrite the defaults
% using explicit options in \includegraphics[width, height, ...]{}
\setkeys{Gin}{width=\maxwidth,height=\maxheight,keepaspectratio}
% Set default figure placement to htbp
\makeatletter
\def\fps@figure{htbp}
\makeatother
\setlength{\emergencystretch}{3em} % prevent overfull lines
\providecommand{\tightlist}{%
  \setlength{\itemsep}{0pt}\setlength{\parskip}{0pt}}
\setcounter{secnumdepth}{5}
\usepackage{booktabs}
\ifLuaTeX
  \usepackage{selnolig}  % disable illegal ligatures
\fi
\usepackage[]{natbib}
\bibliographystyle{plainnat}

\title{R-Blitzkurs}
\usepackage{etoolbox}
\makeatletter
\providecommand{\subtitle}[1]{% add subtitle to \maketitle
  \apptocmd{\@title}{\par {\large #1 \par}}{}{}
}
\makeatother
\subtitle{⚡R}
\author{Sebastian Sauer}
\date{Letzte Aktualisierung: 2022-02-24 12:34:24}

\begin{document}
\maketitle

{
\setcounter{tocdepth}{1}
\tableofcontents
}
\begin{figure}[H]

{\centering \includegraphics[width=1\linewidth]{/Users/sebastiansaueruser/github-repos/__Archiv/stats-illustrations-Allison-Horst/rstats-artwork/rainbowr} 

}

\end{figure}

\href{https://github.com/allisonhorst/stats-illustrations}{Autor: Allison Horst, CC-BY}

\hypertarget{uxfcberblick}{%
\chapter{Überblick}\label{uxfcberblick}}

\hypertarget{tldr}{%
\section{tl;dr}\label{tldr}}

\begin{itemize}
\tightlist
\item
  Einführung in die Statistik-Software R
\item
  keine Vorkenntnisse nötig
\item
  Dauer: 60-90 Min.
\item
  online (Zoom)
\item
  Dozent: Sebastian Sauer
\item
  Vorbereitung: Konto bei RStudio Cloud anlegen
\end{itemize}

\hypertarget{was-sie-hier-lernen-und-wozu-das-gut-ist}{%
\section{Was Sie hier lernen und wozu das gut ist}\label{was-sie-hier-lernen-und-wozu-das-gut-ist}}

R ist die führende Software für Datenanalyse und insofern Eintrittskarte in das Datenzeitalter.
Sie würden gerne R kennenlernen, aber das kostet Sie zu viel Zeit?
Dann kommen Sie zum R-Blitzkurs! In 60 Minuten lernen Sie zentrale Schritte der Datenanalyse.
OK, vielleicht reicht die Zeit auch nur für einen ersten Eindruck -
aber der erste Schritt ist der wichtigste.
Aus dem Inhalt: Daten importieren in R, deskriptive Statistiken erstellen, Diagramme erzeugen, Daten und Ergebnisse exportieren, inferenzstatistische Tests durchführen.
Sie brauchen keine Software vorab installieren; im Kurs wird Ihnen eine Online-Version von R bereitgestellt (kostenlos).
Es ist aber nötig, sich ein Konto dafür vorab anzulegen, bitte lesen Sie die Details in der Kursbeschreibung.

\hypertarget{lernziele}{%
\section{Lernziele}\label{lernziele}}

Nach dem Absolvieren dieses Kurses \ldots{}

\begin{itemize}
\tightlist
\item
  können die Teilnehmis einige zentrale Tätigkeiten der Datenanalyse in R durchführen
\item
  sind die Teilnehmis in der Lage, selber Analysen mit R durchzuführen
\end{itemize}

\hypertarget{inhaltsuxfcberblick}{%
\section{Inhaltsüberblick}\label{inhaltsuxfcberblick}}

\begin{itemize}
\tightlist
\item
  Daten importieren und exportieren
\item
  Deskriptive Statistiken erzeugen
\item
  Datendiagramme erzeugen
\item
  Inferenzstatistische Tests durchführen
\end{itemize}

\hypertarget{allgemeine-hinweise}{%
\section{Allgemeine Hinweise}\label{allgemeine-hinweise}}

\begin{itemize}
\item
  Lesen Sie sich die folgenden Informationen bitte gut durch: \href{https://sebastiansauer.github.io/fopra/Interna/Hinweise.html}{Hinweise}
\item
  Den Quellcode finden Sie \href{https://github.com/sebastiansauer/r-blitzkurs}{in diesem Github-Repo}.
\item
  Sie haben Feedback, Fehlerhinweise oder Wünsche zur Weiterentwicklung? Am besten stellen Sie \href{https://github.com/sebastiansauer/fopra/issues}{hier} einen \emph{Issue} ein.
\item
  Dieses Projekt steht unter der {[}MIT-Lizenz{]}(\url{https://github.com/sebastiansauer/r-blitzkurs/blob/main/LICENSE}
\end{itemize}

\hypertarget{modulliteratur}{%
\chapter{Modulliteratur}\label{modulliteratur}}

Ein Teil der Literatur ist über viele Hochschulbibliotheken als PDF herunterladen;
andere Titel sind offen im Internet verfügbar.
Oft müssen Sie per VPN angemeldet (in der Bib.) sein für Volltextzugriff, wenn Sie nicht auf dem Campus sind.

\hypertarget{lernhilfen}{%
\chapter{Lernhilfen}\label{lernhilfen}}

\href{https://sebastiansauer.github.io/fopra/Interna/Lernhilfen.html}{Lernhilfen}

\hypertarget{modulinhalte}{%
\chapter{Modulinhalte}\label{modulinhalte}}

  \bibliography{book.bib,packages.bib}

\end{document}
